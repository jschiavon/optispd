\documentclass[fleqn]{article}

\usepackage[no-math]{fontspec}
\usepackage[OT1]{eulervm}
\usepackage{microtype,ifthen}
\usepackage{amsmath,amssymb,amsfonts,amsthm,braket,cancel}
\usepackage{titlesec}
\usepackage[usenames,dvipsnames]{xcolor}
\usepackage[style=numeric-comp,backend=biber,doi=false,isbn=false,url=false,date=year]{biblatex}
\usepackage{hyperref}

% Fonts
\defaultfontfeatures{%
    RawFeature={%
        +calt   % *Contextual alternates
        ,+clig  % *contextual ligatures
        ,+ccmp  % *composition & decomposition
        ,+tlig  % 'tex-ligatures': `` '' -- --- !` ?` << >>
        ,+cv06  % narrow guillemets
    }%
}
\setmainfont{EB Garamond}
\newfontfamily{\smallcaps}[RawFeature={+c2sc,+scmp}]{EB Garamond}
\newfontfamily{\swash}[RawFeature={+swsh}]{EB Garamond}

% Sections
\titleformat{\section}{\normalfont \Large \scshape}{\thesection}{1em}{}
\titleformat{\subsection}{\normalfont \large \scshape}{\thesubsection}{1em}{}
\titleformat{\subsubsection}{\normalfont \scshape}{\thesubsubsection}{0.5em}{}
\pagestyle{headings}
\renewcommand{\thesection}{\roman{section}}
\renewcommand{\thesubsection}{\thesection.\roman{subsection}}
\renewcommand{\thesubsubsection}{\thesubsection.\roman{subsubsection}}

% Theorems
\newtheoremstyle{definition}{}{}{\itshape}{\parindent}{\scshape}{.}{1em}{\thmname{#1}\thmnumber{ #2}:}
\newtheoremstyle{theorem}{}{}{}{\parindent}{\scshape}{}{1em}{\thmname{#1}\thmnumber{ #2}:}
\theoremstyle{theorem}
\newtheorem{lemma}{Lemma}[section]
\newtheorem{corollary}[lemma]{Corollary}
\theoremstyle{definition}
\newtheorem{definition}{Definition}

% Math operators
\DeclareMathOperator*{\argmin}{argmin}
\DeclareMathOperator*{\argmax}{argmax}
\DeclareMathOperator{\logm}{Log}
\newcommand{\norm}[2][]{\left\Vert#2\right\Vert_{#1}}
\newcommand{\abs}[1]{\left\vert#1\right\vert}
\newcommand{\SN}[2][]{\mathcal{SN}_{#1}\left(#2\right)}

% Metadata
\title{Skewnormal}
\newcommand{\myName}{Jacopo Schiavon}
\newcommand{\myMail}{\href{mailto:jschiavon@stat.unipd.it}{\ttfamily jschiavon@stat.unipd.it}}
\newcommand{\myDept}{Department of Statistical Sciences, University of Padova}
\author{\myName\thanks{\myDept. Contact: \myMail}}
\hypersetup{pdfauthor={\myName},
    pdfcreator={\myName},
    breaklinks=True,
    colorlinks=true,       	% false: boxed links; true: colored links
    linkcolor=MidnightBlue, % color of internal links
    citecolor=ForestGreen,	% color of links to bibliography
    filecolor=Plum,			% color of file links
    urlcolor=Sepia			% color of url link
}

% Bibliography
\setlength\bibitemsep{1.5\itemsep}
\setlength\bibhang{1.5\parindent}
\renewcommand*{\mkbibnamefamily}[1]{\textsc{#1}}
\renewcommand*{\mkbibnamegiven}[1]{\textsc{#1}}
\renewcommand*{\mkbibnameprefix}[1]{\textsc{#1}}
\renewcommand*{\mkbibnamesuffix}[1]{\textsc{#1}}
\renewcommand*{\labelnamepunct}{\par}
\addbibresource{biblio.bib}

\setlength{\parindent}{0pt}

\begin{document}
    \maketitle

    \section{Parametrization}
    Let $y\in\mathbb{R}^d$, we say that $y\sim\SN[d]{\xi, \bar\Sigma, \delta}$ if
    \begin{equation}\label{eq:first}
        p(y\mid \xi, \bar\Sigma, \delta) = \int_0^\infty 2\phi_{d+1}\left(\left[y^\top, z\right]^\top\mid \mu, \Omega\right)dz
    \end{equation}
    with $\mu=\left[\xi^\top, 0\right]^\top$ and $\Omega=\begin{pmatrix}
        \omega\Sigma\omega  &   \omega\delta\\
        \delta^\top\omega   &   1
    \end{pmatrix}$ and $\bar\Sigma = \omega\Sigma\omega$ is the decomposition of the covariance matrix in correlation matrix and the diagonal matrix with variances.

    By defining $\theta= \omega\delta$ and $\Psi = \bar\Sigma - \theta\theta^\top$, we can rewrite the previous density as
    \begin{equation}\label{eq:second}
        p(y\mid \xi, \bar\Sigma, \delta) \propto \int_0^\infty 2\phi_1(z)\phi_d\!\left(y\mid \xi+\theta z, \Psi\right)dz.
    \end{equation}
    Note that $\abs{\Omega} = \abs{\omega\Sigma\omega - \omega\delta\delta^\top\omega} = \abs{\omega\left(\Sigma-\delta\delta^\top\right)\omega} = \abs{\Psi}$ and
    \begin{equation*}
        \Omega^{-1} = \begin{pmatrix}
            \Psi^{-1}   &   -\Psi^{-1}\theta\\
            - \theta^\top\Psi^{-1}  &   1 + \theta^\top\Psi^{-1}\theta
        \end{pmatrix}
    \end{equation*}

    Moreover, by rearranging the terms from equation~\eqref{eq:second} we can write:
    \begin{align*}
        p(y\mid \xi, \bar\Sigma, \delta) &\propto \int_0^\infty 2\phi_1(z\mid\bar\mu,\bar\sigma^2) \phi_d\!\left(y\mid \xi, \Psi\right) \exp\left[\frac{\bar\mu^2}{2\bar\sigma^2}\right]dz\\
        &= \phi_d\!\left(y\mid \xi, \Psi\right) \exp\left[\frac{\bar\mu^2}{2\bar\sigma^2}\right] 2 \int_{-\bar\mu/\bar\sigma}^\infty\phi_1(z)dz\\
        &= 2\phi_d\!\left(y\mid \xi, \Psi\right) \exp\left[\frac{\bar\mu^2}{2\bar\sigma^2}\right]\Phi_1\!\left(\frac{\bar\mu}{\bar\sigma}\right)\\
        &= 2\phi_d\!\left(y\mid \xi, \Psi\right) \exp\left[\frac{1}{2}(y-\xi)^\top\alpha\alpha^\top(y-\xi)\right]\Phi_1\!\left(\alpha^\top(y-\xi)\right)\\
        &= 2\phi_d\!\left(y\mid \xi, \Psi - \alpha\alpha^\top\right) \Phi_1\!\left(\alpha^\top(y-\xi)\right)
    \end{align*}
    where we have used
    \begin{align*}
        \bar\mu &= \frac{(y-\xi)^\top\Psi^{-1}\theta}{1 + \theta^\top\Psi^{-1}\theta}        &       \bar\sigma^2 &= \left(1 + \theta^\top\Psi^{-1}\theta\right)^{-1}
    \end{align*}
    and we defined
    \begin{equation*}
        \alpha = \frac{\Psi^{-1}\theta}{\sqrt{1 + \theta^\top\Psi^{-1}\theta}}
    \end{equation*}

    \section{Constraints}
    In order for $\Psi$ (and thus $\Omega$) to be positive definite, a constrain should be put on $\delta$ and $\bar\Sigma$.  First of all, recall that the matrix $\theta\theta^\top$ has only one strictly positive eigenvalue, equal to $\norm{\theta}^2$, while all the others are 0. As it can be proven that $\Psi$ is SPD if and only if the smallest eigenvalue of $\bar\Sigma$ is larger than $\norm{\theta}^2$, we can require that
    \begin{equation*}
        \norm{\theta}^2 = \delta^\top\omega\omega\delta \leq \min_i\lambda_i(\bar\Sigma)
    \end{equation*}



    \section{Data generation mechanism}
    To generate samples from a skewnormal we exploit equation~\eqref{eq:first} and we proceed in the following way:
    \begin{itemize}
        \item We compute $\mu$ and $\Omega$ from the parameters $\xi$, $\bar\Sigma$ and $\delta$
        \item We generate a sample from a $(d+1)$-variate normal distribution: $Z \sim \mathcal{N}_{d+1}(\mu,\Omega)$
        \item if $Z[d+1] \geq 0$ then $y = Z[:d]$, else $y = - Z[:d]$.
    \end{itemize}



    \printbibliography



\end{document}